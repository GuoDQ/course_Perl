\documentclass{TIJMUjiaoanSY}
\pagestyle{empty}

\begin{document}

\kecheng{分子生物计算}
\shiyan{实验5 \ 子程序和Bugs}
\jiaoshi{伊现富}
\zhicheng{讲师}
\riqi{2015年10月12日10:00-12:00}
\duixiang{生物医学工程与技术学院2013级生信班(本)}
\renshu{28}
\leixing{验证型}
\fenzu{一人一机}
\xueshi{2}
\jiaocai{Perl语言在生物信息学中的应用——基础篇}

\firstHeader
\maketitle
\thispagestyle{empty}

\mudi{
\begin{itemize}
  \item 了解Perl语言中的模块和子程序库。
  \item 熟悉Perl调试器。
  \item 掌握Perl语言中的子程序。
\end{itemize}
}

\fenpei{
\begin{itemize}
  \item (10')子程序:回顾Perl语言中子程序的定义、使用等基本知识。
  \item (5')程序调试:总结调试Perl程序的方法。
  \item (85')实验操作:编写用于生物序列数据处理的子程序并进行调试。
\end{itemize}
}

\cailiao{
\begin{itemize}
  \item 主要仪器:一台安装有Perl语言(Linux操作系统)的计算机。
\end{itemize}
}
\zhongdian{
\begin{itemize}
  \item 重点难点:传递数据给子程序的方法;Perl调试器的使用。
  \item 解决策略:通过演示进行学习,通过练习熟练掌握。
\end{itemize}
}

\sikao{
\begin{itemize}
  \item 比较传递数据给子程序的方法。
  \item 总结调试Perl程序的方法。
\end{itemize}
}

\cankao{
\begin{itemize}
  \item Beginning Perl for Bioinformatics, James Tisdall, O'Reilly Media, 2001.
  \item Perl语言入门(第六版),Randal L. Schwartz, brian d foy \& Tom Phoenix著,盛春\ 译,东南大学出版社,2012。
  \item Mastering Perl for Bioinformatics, James Tisdall, O'Reilly Media, 2003.
  \item 维基百科等网络资源。
\end{itemize}
}

\firstTail

\newpage
\otherHeader

\begin{enumerate}
  \item 子程序(10分钟)
\vspace*{-1em}
\begin{multicols}{3}
    \begin{enumerate}
      \item 定义:sub
      \item 调用:\&
      \item 返回值:reture
      \item 收集参数:\verb|@_|
      \item 作用域:my
      \item 传递数据:值 vs. 引用
    \end{enumerate}
\end{multicols}
\vspace*{-1em}
  \item 程序调试(5分钟)
\vspace*{-1em}
\begin{multicols}{2}
    \begin{enumerate}
      \item use warnings;和use strict;
      \item 选择性注释
      \item 添加print语句
      \item Perl调试器:p,n,s,v,c,b,B,w,R...
    \end{enumerate}
\end{multicols}
\vspace*{-1em}
  \item 实验操作(85分钟)
    \begin{enumerate}
      \item 编写子程序
\begin{verbatim}
#!/usr/bin/perl -w

$dna = 'CGACGTCTTCTCAGGCGA';
$longer_dna = addACGT($dna);
print "I added ACGT to $dna and got $longer_dna\n\n";

sub addACGT {
    my ($dna) = @_;
    $dna .= 'ACGT';
    return $dna;
}
\end{verbatim}
      \item 使用my限定作用域
\begin{verbatim}
#!/usr/bin/perl -w

$dna = 'AAAAA';
$result = A_to_T($dna);
print "I changed all the A's in $dna to T's and got $result\n\n";

sub A_to_T {
    my ($input) = @_;
    my $dna = $input;
    $dna =~ s/A/T/g;
    return $dna;
}
\end{verbatim}
      \item 传递数据给子程序
	\begin{enumerate}
	  \item 通过值传递
\begin{verbatim}
#!/usr/bin/perl -w
use strict;

my $i = 2;
simple_sub($i);
print "In main program, after the subroutine call, \$i equals $i\n\n";

sub simple_sub {
    my($i) = @_;
    $i += 100;
    print "In subroutine simple_sub, \$i equals $i\n\n";
}
\end{verbatim}

\otherTail
\newpage
\otherHeader

	  \item 通过引用传递
\begin{verbatim}
#!/usr/bin/perl

use strict;
use warnings;

my @i = ('1', '2', '3');
my @j = ('a', 'b', 'c');
print "In main program before calling subroutine: i = " .  "@i\n";
print "In main program before calling subroutine: j = " .  "@j\n";
reference_sub(\@i, \@j);
print "In main program after calling subroutine: i = " .  "@i\n";
print "In main program after calling subroutine: j = " .  "@j\n";
exit;

sub reference_sub {
    my($i, $j) = @_;
    print "In subroutine : i = " . "@$i\n";
    print "In subroutine : j = " . "@$j\n";
    push(@$i, '4');
    shift(@$j);
}
\end{verbatim}
	\end{enumerate}
      \item 使用命令行参数
\begin{verbatim}
#!/usr/bin/perl -w

use strict;

my ($USAGE) = "$0 DNA\n\n";
unless (@ARGV) {
    print $USAGE;
    exit;
}
my ($dna) = $ARGV[0];
my ($num_of_Gs) = countG($dna);
print "\nThe DNA $dna has $num_of_Gs G\'s in it!\n\n";
exit;

sub countG {
    my ($dna) = @_;
    my ($count) = 0;
    $count = ( $dna =~ tr/Gg// );
    return $count;
}
\end{verbatim}

\otherTail
\newpage
\otherHeader

      \item 使用Perl调试器修复Bugs\textcolor{red}{(注意:程序中的bug不止一个)}
\begin{verbatim}
#!/usr/bin/perl

my $dna = 'CGACGTCTTCTAAGGCGA';
my @dna;
my $receivingcommittment;
my $previousbase = '';
my $subsequence = '';

if (@ARGV) {
    my $subsequence = $ARGV[0];
}
else {
    $subsequence = 'TA';
}

my $base1 = substr( $subsequence, 0, 1 );
my $base2 = substr( $subsequence, 1, 1 );

@dna = split( '', $dna );

foreach (@dna) {
    if ($receivingcommittment) {
        print;
        next;
    }
    elsif ( $previousbase eq $base1 ) {
        if (/$base2/) {
            print $base1, $base2;
            $recievingcommitment = 1;
        }
    }
    $previousbase = $_;
}

print "\n";
exit;
\end{verbatim}
    \end{enumerate}
\end{enumerate}

\otherTail


\end{document}
