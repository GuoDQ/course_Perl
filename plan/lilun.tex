\documentclass{TIJMUjiaoanLL}
\pagestyle{empty}

\begin{document}

\kecheng{分子生物计算}
\neirong{ \ / 第章}
\jiaoshi{伊现富}
\zhicheng{讲师}
\riqi{2015年月日8:00-10:00}
\duixiang{生物医学工程与技术学院2013级生信班(本)}
\renshu{28}
\fangshi{理论讲授}
\xueshi{2}
\jiaocai{Perl语言在生物信息学中的应用——基础篇}

\firstHeader
\maketitle
\thispagestyle{empty}

\mudi{
\begin{itemize}
  \item 掌握:。
  \item 熟悉:。
  \item 了解:。
  \item 自学:。
\end{itemize}
}

\fenpei{
\begin{itemize}
  \item (5')引言与导入:。
  \item 
  \item 
  \item 
  \item (5')总结与答疑:总结授课内容中的知识点与技能,解答学生疑问。
\end{itemize}
}

\zhongdian{
\begin{itemize}
  \item 重点:。
  \item 难点:。
  \item 解决策略:通过实例演示帮助学生理解、记忆。
\end{itemize}
}

\waiyu{
\vspace*{-10pt}
\begin{multicols}{2}

\end{multicols}
\vspace*{-10pt}
}

\fuzhu{
\begin{itemize}
  \item 多媒体:。
  \item 板书:。
  \item 演示:。
\end{itemize}
}

\sikao{
\vspace*{-10pt}
\begin{multicols}{2}
\begin{itemize}
  \item 
  \item 
  \item 
\end{itemize}
\end{multicols}
\vspace*{-10pt}
}

\cankao{
\begin{itemize}
  \item Beginning Perl for Bioinformatics, James Tisdall, O'Reilly Media, 2001.
  \item Perl语言入门(第六版),Randal L. Schwartz, brian d foy \& Tom Phoenix著,盛春\ 译,东南大学出版社,2012。
  \item Mastering Perl for Bioinformatics, James Tisdall, O'Reilly Media, 2003.
  \item 维基百科等网络资源。
\end{itemize}
}

\firstTail

\newpage
\otherHeader

\begin{enumerate}
  \item 引言与导入(5分钟)
  \item 
  \item 

\otherTail
\newpage
\otherHeader

  \item 
  \item 总结与答疑(5分钟)
    \begin{enumerate}
      \item 知识点
	\begin{itemize}
	  \item 
	  \item 
	  \item 
	\end{itemize}
      \item 技能
	\begin{itemize}
	  \item 
	  \item 
	  \item 
	\end{itemize}
    \end{enumerate}
\end{enumerate}

\otherTail

%\parpic[fr]{\includegraphics[width=\textwidth]{}}

\end{document}
